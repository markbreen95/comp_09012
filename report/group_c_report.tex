% ********************************************************************
% *                  Format for IMVIP 2019  papers,                  *
% *         based on the IMVIP 2001, 2006, 2014-2018 templates       *
% ********************************************************************
\documentclass[a4paper,11pt]{article}



\setlength{\topmargin}{-0.5cm}
\setlength{\headsep}{.5cm}
%\setlength{\footskip}{1.0cm}
\setlength{\textheight}{24cm}
\setlength{\textwidth}{17cm}
\setlength{\evensidemargin}{-.5cm}
\setlength{\oddsidemargin}{-.5cm}



\usepackage{fourier}
\usepackage{color}
 \usepackage{graphicx}
\usepackage{url}
\usepackage[affil-it]{authblk}
\usepackage{amsmath}
\usepackage{wrapfig}
\usepackage{xspace}

\usepackage{booktabs}

\usepackage[T1]{fontenc}
\usepackage{times}


\pagestyle{empty}

%%%%
\begin{document}

\title{COMP09012 - Machine Learning - Group Project}

\author{Mark Breen, Phillip Garrad, Gerard Comerford}
\affil{Institute of Technology, Sligo}
\date{}
\maketitle
\thispagestyle{empty}



\begin{abstract}
There are hundreds of inputs which are important to consider when making decisions in autonomous vehicles. The ability to quantify these factors is of the upmost importance since quantitative data is ultimately what will need to be fed into any decision making algorithm. Three of these factors will be considered in this paper. Three such factors: driving style, road surface condition and traffic are investigated in this paper.
\end{abstract}
\textbf{Keywords:} Imaging, Image Processing, Machine Vision, etc. (Maximum five)



%%%%%%%%%%%%%%%%%%%%%%
\section{Introduction}

The goal of this paper is to predict the value of multiple classes relating to the road surface, traffic congestion and driving style of a vehicle.


%%
\section{Methods}

This section will highlight the methods used for data cleaning and preparation, along with a description of the machine learning models and metrics we used.

\subsection{Data}

Data was received in the form of four text file of comma separated value (CSV) extension. The delimiter encoded in the four CSV files is a semi-colon (;). The data contains information recorded for two trips in two vehicles, the vehicles being an Opel Corsa 1.3 HDi (95 CV) and Peugeot 207 1.4 HDi (70 CV). Since there is two recorded trips per vehicle, this means there is a total of four CSV data files. The data was recorded from the vehicle on-board diagnostics (OBD) and micro-devices embedded in the smartphone of the driver operating the vehicle at the time.

A breakdown of the number of rows in each vehicle dataset can be found in Table \ref{tab:file_rows}

\begin{table}[!ht]
	\caption{Number of rows in each of the raw data files}\label{tab:file_rows}
	\centering
	\begin{tabular}{lccccc}\toprule
		Filename & Vehicle & Number of rows & Number of columns \\ 
		opel\_corsa\_01.csv & Opel Corsa 1.3 HDi (95 CV) & 7392 & 17 \\ 
		opel\_corsa\_02.csv & Opel Corsa 1.3 HDi (95 CV) & 4328 & 17 \\ 
		peugeot\_207\_01.csv & Peugeot 207 1.4 HDi (70 CV) & 8614 & 17 \\ 
		peugeot\_207\_02.csv & Peugeot 207 1.4 HDi (70 CV) & 4623 & 17 \\ \bottomrule
	\end{tabular}
\end{table}

A description of the seventeen different columns available in the dataset is shown in Table \ref{tab:column_types}

\begin{table}[!ht]
	\caption{Data column descriptions}\label{tab:column_types}
	\centering
	\begin{tabular}{lccccc}\toprule
		Column Name & Column Description & Data Type \\ 
		AltitudeVariation & The altitude change calculated over 10 seconds & Decimal \\ 
		VehicleSpeedInstantaneous & The current speed of the vehicle & Decimal \\ 
		VehicleSpeedAverage & Vehicle average speed in the last 60 seconds & Decimal \\ 
		VehicleSpeedVariance & Speed variance in the last 60 seconds & Decimal \\ 
		VehicleSpeedVariation & Speed variation for every second of detection & Decimal \\ 
		LongitudinalAcceleration & Longitudinal acceleration & Decimal \\ 
		EngineLoad & Engine load as a percentage & Decimal \\ 
		EngineCoolantTemperature & The engine coolant temperature in degrees celsius & Decimal \\ 
		ManifoldAbsolutePressure & Manifold air pressure & Decimal \\ 
		EngineRPM & Revolutions Per Minute (RPM) of the engine & Decimal \\ 
		MassAirFlow & Mass Air Flow measured in grams per second & Decimal \\ 
		IntakeAirTemperature & Intake air temperature at the engine entrance & Decimal \\ 
		VerticalAcceleration & Vertical acceleration & Decimal \\ 
		FuelConsumptionAverage & Average fuel consumption in litres per 100 km & Decimal \\ 
		roadSurface & Road surface condition & String \\ 
		traffic & Traffic congestion condition & String \\ 
		drivingStyle & Driving style & String \\ \bottomrule
	\end{tabular}
\end{table}

\begin{table}[!ht]
	\caption{Percentage values missing in each column as a proportion of the number of rows of each dataset}\label{tab:column_types}
	\centering
	\begin{tabular}{lccccc}\toprule
		Column Name & opel\_corsa\_01.csv & opel\_corsa\_2.csv & peugeot\_207\_1.csv & peugeot\_207\_2.csv \\ 
		AltitudeVariation & 0.73\% & 0.83\% & 0.73\% & 0.58\% \\ 
		VehicleSpeedInstantaneous & 0.09\% & 0.18\% & 0.10\% & 0.43\% \\ 
		VehicleSpeedAverage & 4.79\% & 5.45\% & 4.82\% & 3.83\% \\ 
		VehicleSpeedVariance & 4.79\% & 5.45\% & 4.82\% & 3.83\% \\ 
		VehicleSpeedVariation & 0.46\% & 0.88\% & 0.91\% & 0.82\% \\ 
		LongitudinalAcceleration & 0.00\% & 0.00\% & 0.00\% & 0.00\% \\ 
		EngineLoad & 0.00\% & 0.00\% & 0.06\% & 0.00\% \\ 
		EngineCoolantTemperature & 0.00\% & 0.00\% & 0.06\% & 0.00\% \\ 
		ManifoldAbsolutePressure & 0.00\% & 0.00\% & 0.06\% & 0.00\% \\ 
		EngineRPM & 0.00\% & 0.00\% & 0.06\% & 0.00\% \\ 
		MassAirFlow & 0.00\% & 0.00\% & 0.06\% & 0.00\% \\ 
		IntakeAirTemperature & 0.00\% & 0.00\% & 0.06\% & 0.00\% \\ 
		VerticalAcceleration & 0.00\% & 0.00\% & 0.00\% & 0.00\% \\ 
		FuelConsumptionAverage & 1.31\% & 1.20\% & 1.11\% & 0.89\% \\ 
		roadSurface & 0.00\% & 0.00\% & 0.00\% & 0.00\% \\ 
		traffic & 0.00\% & 0.00\% & 0.00\% & 0.00\% \\ 
		drivingStyle & 0.00\% & 0.00\% & 0.00\% & 0.00\% \\ \bottomrule
	\end{tabular}
\end{table}

Due to large variations in the value and units of measurement across the numerical columns it was decided to scale the numerical columns to be within the range $(0,1)$. To do this equation was applied column-wise for each column $j$ in the numerical subset of the data columns:

\begin{equation}
	X_{j \text{ scaled}} = \frac{X_{j}-\text{min}(X_{j})}{\text{max}(X_{j})-\text{min}(X_{j})}
\end{equation}

\subsection{Classification Models}

\subsubsection{Random Forest Classifier}

A random forest classifier works on the basis of decision trees and is essentially an ensemblement of decision tree classifiers. A random forest algorithm works drawing bootstrapped samples from the original dataset and fitting a decision tree using a random sample of the predictors. In order to predict using new data, an aggregate prediction is calculated from $n_{tree}$, the number of trees. A majority concensus is what's used to determine the predicted classification label.

\subsubsection{K Neighbours Classifier}

The K-Nearest Neighbours algorithm differs from the Random Forest Classifier in that instead of inferring a model from the data, the data is used directly. The K-Nearest Neighbours algorithm essentially works as follows (for a hyper-parameter $k$):

\begin{itemize}
	\item Retrieve the $k$ most observations
	\item Take the mode of $k$ votes to determine which class the observation belongs to
\end{itemize}

\subsection{Model Metrics}

The $F_{1}$ score was used to assess the accuracy of the model. The reason why the $F_{1}$ score has been chosen as a model metric here is due to imbalanced classes being observed in the dependent variables, meaning that the number of false positives and false negatives is crucial to consider when comparing models. The $F_{1}$ score is defined as the harmonic mean of the precision and recall. Precision is defined as:

\begin{equation}
	\text{Precision} = \frac{ \text{Number of true positives} }{\text{Number of true positives} + \text{Number of false positives}}
\end{equation}

Recall is defined as:

\begin{equation}
	\text{Precision} = \frac{ \text{Number of true positives} }{\text{Number of true positives} + \text{Number of false negatives}}
\end{equation}

Then, the $F_{1}$ score is defined as:

\begin{equation}
	F_{1} = 2 \times \frac{\text{precision} \times \text{recall}}{\text{precision} + \text{recall}}
\end{equation}

Since the $F_{1}$ score is calculated on a per-class basis and there are multiple classes in two of the dependent variables, an average $F_{1}$ score was calculated to in order to assess model accuracy.

\section{Results}

\subsection{Random Forest Classifier}

\subsection{K Neighbours Classifier}

\section*{Acknowledgments}

The Acknowledgments section, if included,
		       follows the main body of the text and is headed
		       ``Acknowledgments,'' printed in the same style
		       as a section heading, but without a number.

\bibliographystyle{apalike}

\bibliography{imvip}


\end{document}

